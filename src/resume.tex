\documentclass{resume}
\usepackage[left=0.75in,top=0.6in,right=0.75in,bottom=0.6in]{geometry}
\newcommand{\tab}[1]{\hspace{.2667\textwidth}\rlap{#1}}
\newcommand{\itab}[1]{\hspace{0em}\rlap{#1}}
\name{Nasif Mauthoor} % Your name
\address{North of the Wall, Outside the reach of every kingdom}
\address{(+0)9999999999 \\ nasifmrmonline@gmail.com}

\begin{document}

    \begin{rSection}{}

        \begin{center}
            A bright Computer Engineering student with aspirations in
            robotics and software development.
        \end{center}

    \end{rSection}

    \begin{rSection}{Education}

        \textbf{University of Guelph, Canada}
        \hfill{\em September 2019 - Present}
        \\ Bachelor of Engineering, Computer Engineering.
        \hfill{Overall: TBD - Current GPA: 3.0}

        \textbf{École Secondaire Jeunes Sans Frontières}
        \hfill {\em September 2019 - Present}
        \\ French High School Diploma.
        International Baccalaureate.
        \hfill { Overall GPA: 3.9}

    \end{rSection}

    \begin{rSection}{Significant Projects}

        A full list of other notable projects and works with full descriptions
        and documentation can be found on my GitHub page, which is continuously
        being updated (github.com/MoonJumpMania).

        \textbf{Bash Inspired Custom Shell}
        \\ A computer shell that can functionally do the basics of what the GNU
        Bash shell can like execute commands, piping IO and running processes in
        the background.
        Solution written in \textbf {C}.

        \textbf{CPU and Memory Scheduling Simulator}
        \\ A Discrete Event Scheduler that can manage the execution of threads
        and processes using various scheduling algorithms such as Round Robin and
        First-Come-First-Serve.
        The Memory scheduler kicks processes in and out of memory in method of
        choice between algorithms such as Best-Fit, Worst-Fit, First-Fit and
        Next-Fit.
        Both Schedulers were written in \textbf {C}.

        \textbf{VHDL CPU Simulator}
        \\ A basic virtual CPU that takes advantage of registers, memory and
        multiplexers to execute \textbf{Mars MIPS} assembly commands store data
        in memory and registers.
        Written in \textbf{VHDL}.

        \textbf{Cellular Automata}
        \\ A recreation Conway's Game of Life.
        Later used this method to create an algorithm to generate randomized
        terrain generation.
        Written in \textbf{C}.

        \textbf{Fluid Physics Simulator}
        \\ A simulation of liquids moving on different surfaces in a 2D space.
        Currently still working on a final result.
        Written in \textbf{C\#} using the \textbf{Unity} game engine.

        \textbf{Ludum Dare Game Jams}
        \\ A game jam where I had to build games in one single weekend.
        Created a game to teach about ecology and the pairing of animal species.
        Created a second game showing how a 3D space can be visualised in two
        dimensions.
        Worked alone and as a team.
        Both written in \textbf{C\#} using the \textbf{Unity} game engine.

        \textbf{Colossal Cave Adventure}
        \\ A recreation of Will Crowther's game from 1976.
        Written in \textbf{Java} using the \textbf{Swing} framework.

        \textbf{Discord Utility Bot}
        \\ A Discord application that can run on any Discord server to play
        music, read and write suggestions from and to files and do some basic
        automation.
        Written in \textbf{Python} using the \textbf{discord.py} API\@.

        \textbf{Basketball Playing Robot}
        \\ A school project at University of Guelph called Teddy Bear Wheelchair.
        The wheelchair robot needs to carry around a teddy bear and have it
        shoot ping-pong balls at a basketball net with precise accuracy and have
        it run a course designed to test its mobility.
        Written in \textbf{C++} powered by \textbf{Arduino}.

    \end{rSection}

    \pagebreak

    \begin{rSection}{Technical Strengths}

        \begin{tabular}{ @{} >{\bfseries}l @{\hspace{6ex}} l }
            Programming Languages: \    & C, C++, Python, Java, C\#, Matlab, R, Lua, VHDL           \\\\

            Tools:                      & Visual Studio/VSCode, LaTeX, Git, IntelliJ, Photoshop,    \\
                                        & Arduino, Solidworks, AutoCAD, Xilinx Vivado, FPGA, Maven  \\\\

            Frameworks and Engines:     & Unity, Java-Swing, Scratch, .NET                          \\\\

            Libraries:                  & discord.py, Arduino                                       \\\\
        \end{tabular}

    \end{rSection}

    \begin{rSection}{Work Experience}

        \begin{rSubsection}{Cloud Times Global \& Bright Learning Center}{2022 - Current}{Computer Science Teacher}{}
            \item Taught children and teenagers the fundamentals of programming and computer science.
            \item Organized educational projects and exercises to test student comprehension.
            \item Created documentation and resources from past teacher's courses and from course outlines.
            \item Used \textbf{Python}, \textbf{Java}, \textbf{C\#/Unity} and \textbf{Scratch}.
        \end{rSubsection}

        \begin{rSubsection}{Reliance Home Comfort}{2019 - 2021}{Customer Service Agent}{}
            \item Answered calls from customers who had issues with their home appliances.
            \item Booked appropriate technicians and plumbers to help customers with their issues.
            \item Worked with other departments relating to billing and shipping products.
        \end{rSubsection}

        \begin{rSubsection}{Microsoft - Camp 21}{2016 - 2019}{Event Leader}{}
            \item Empowered the youth to explore new technologies and show them what is possible nowadays.
            \item Taught 3D printing, 3D modeling and IOT device programming.
            \item Worked with other event leaders to plan out various workshops to keep kids engaged.
        \end{rSubsection}

    \end{rSection}

    \begin{rSection}{Personal Notes}
        \item I enjoy tinkering with electronics and building desktop computers.
        I code multiple little applications that help me with everyday tasks such
        as answering calls and sending messages.
        \item I can fluently communicate in English and French.
    \end{rSection}

    \begin{rSection}{References}
        References available upon request.
        Please contact me by email preferably or by phone.
    \end{rSection}

\end{document}
